	\section{Moto rettilineo}

\paragraph{Versore $\hat \tau$}
 Vettore tangente alla traiettoria nel punto di ascissa (curvilinea) $s$, di modulo unitario 
perché $\frac{\Delta \overset{\to}{r}}{\Delta s} \to 1$ per $\Delta s \to 0$ (quindi è un versore).
E' definito come: \begin{equation} 
	\hat \tau \overset{\text {def}}{=}\lim _{\Delta s\to 0}\frac{\overset{\to}{r}(s+\Delta s)-\overset{\to}{r}(s)}{\Delta s}
\end{equation}




\paragraph{Velocità scalare} 
La velocità scalare indica direzione e verso con cui un punto materiale $P$ percorre la traiettoria. 
Sia $s$ l'ascissa curvilinea, allora la velocità scalare di un punto materiale nel tempo è data da: 
\begin{equation}
	v_s(t)\overset{\text {def}}{=}\frac{ds(t)}{dt}\equiv \overset{\cdot}{s}(t)
\end{equation}


 
