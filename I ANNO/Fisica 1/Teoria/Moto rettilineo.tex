\section{Moto rettilineo}



\subsection{Legge oraria e traiettoria}
La traiettoria di un punto materiale è una curva $\gamma$ costituita dall'insieme delle posizioni occupate dal punto 
nei diversi istanti, cioè: 
\begin{equation} 
\begin{cases}x=x(t)\\ y=y(t) \\ z=z(t)\end{cases}
\end{equation}
La \textbf{legge oraria} specifica come è percorsa la traiettoria.

\paragraph{Ascissa curvilinea}
E' un valore reale $s$, positivo o negativo a seconda del verso di orientazione della curva $\gamma$, che indica la distanza del 
punto materiale $P$ dall'origine fissato.



\paragraph{Versore $\hat \tau$}
 Vettore tangente alla traiettoria nel punto di ascissa (curvilinea) $s$, di modulo unitario 
perché $\frac{\Delta \overset{\to}{r}}{\Delta s} \to 1$ per $\Delta s \to 0$ (quindi è un versore).
E' definito come: \begin{equation} 
	\hat \tau \overset{\text {def}}{=}\lim _{\Delta s\to 0}\frac{\overset{\to}{r}(s+\Delta s)-\overset{\to}{r}(s)}{\Delta s}
\end{equation}


\paragraph{Velocità scalare} 
La velocità scalare indica direzione e verso con cui un punto materiale $P$ percorre la traiettoria. 
Sia $s$ l'ascissa curvilinea, allora la velocità scalare di un punto materiale nel tempo è data da: 
\begin{equation}
	v_s(t)\overset{\text {def}}{=}\frac{ds(t)}{dt}\equiv \overset{\cdot}{s}(t)
\end{equation}


 
