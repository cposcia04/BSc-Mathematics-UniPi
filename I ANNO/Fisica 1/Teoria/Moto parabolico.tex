\section{Moto parabolico}

\subsection{Equazione della traiettoria di un moto parabolico}

Nel moto parabolico la velocità iniziale ha due componenti: \[v_{x0}=v_0\cos \vartheta\qquad v_{y0}=v_0\sin \vartheta\]
Le componenti $x,y$ della legge oraria sono: \[x(t)=v_0\cos \vartheta \cdot t \qquad y(t)=v_0\sin \vartheta \cdot t - \frac{1}{2}gt^2 \]
Isolando la variabile tempo da $x(t)=v_0\cos \vartheta \cdot t$ si ottiene $t=\frac{x}{v_0 \cos \vartheta}$.
Sostituisco $t=\frac{x}{v_0 \cos \vartheta}$ in $y(t)$: 
\begin{equation} 
y(x)=v_0 \sin\vartheta \frac{x}{v_0\cos\vartheta}-\frac{1}{2}g \frac{x^2}{(v_0\cos \vartheta)^2 }=(\tan \vartheta)x-\frac{g}{2(v_0\cos \vartheta)^2}x^2
\end{equation}

La gittata massima si ha quando l'angolo di inclinazione rispetto al riferimento orizzontale è $\frac{\pi}{4}$
