\section{Foglio 1}

	

	\subsection{Esercizio 1} Sia \(\mathbb{R}^+ = \{x \in \mathbb{R} \mid x > 0\}\). Definiamo su \(\mathbb{R}^+\) l'operazione \(\oplus : \mathbb{R}^+ \times \mathbb{R}^+ \to \mathbb{R}^+\), \(a \oplus b = a \cdot b\) per ogni \(a, b \in \mathbb{R}^+\), ed il prodotto esterno \(\odot : \mathbb{R} \times \mathbb{R}^+ \to \mathbb{R}^+\), \(\lambda \odot a = a^\lambda\) per ogni \(\lambda \in \mathbb{R}\), \(a \in \mathbb{R}^+\).
	
	
	a) Verificare che \((\mathbb{R}^+, \oplus, \odot)\) è uno spazio vettoriale su \(\mathbb{R}\) e determinarne la dimensione. \\
	b) Determinare un isomorfismo tra \((\mathbb{R}^+, \oplus, \odot)\) ed un opportuno \(\mathbb{R}^k\). \\
	
	\paragraph{Soluzione}
	a1) Per dimostrare che \((\mathbb{R}^+, \oplus, \odot)\) è uno spazio vettoriale su \(\mathbb{R}\), dobbiamo verificare che le seguenti proprietà siano soddisfatte: \\
	
	\textbf{1. Chiusura dell'operazione \(\oplus\):}
	\[
	a \oplus b = a \cdot b \quad \text{è un elemento di } \mathbb{R}^+ \text{ per ogni } a, b \in \mathbb{R}^+.
	\]
	Chiaramente, il prodotto di due numeri positivi è ancora un numero positivo, quindi \(\mathbb{R}^+\) è chiuso rispetto a \(\oplus\). \\
	
	\textbf{2. Commutatività di \(\oplus\):}
	\[
	a \oplus b = b \oplus a \quad \text{per ogni } a, b \in \mathbb{R}^+.
	\]
	Il prodotto di numeri reali è commutativo, quindi questa proprietà è soddisfatta. \\
	
	\textbf{3. Associatività di \(\oplus\):}
	\[
	(a \oplus b) \oplus c = a \oplus (b \oplus c) \quad \text{per ogni } a, b, c \in \mathbb{R}^+.
	\]
	Anche questa proprietà è soddisfatta poiché il prodotto di numeri reali è associativo. \\
	
	\textbf{4. Esistenza dell'elemento neutro per \(\oplus\):}
	L'elemento neutro per \(\oplus\) deve essere un elemento \(e \in \mathbb{R}^+\) tale che \(a \oplus e = a\) per ogni \(a \in \mathbb{R}^+\). Dato che \(a \oplus 1 = a \cdot 1 = a\), l'elemento neutro è \(e = 1\). \\
	
\textbf{	5. Esistenza dell'elemento opposto per \(\oplus\):}
	Per ogni \(a \in \mathbb{R}^+\), esiste un \(b \in \mathbb{R}^+\) tale che \(a \oplus b = 1\). Questo è dato da \(b = \frac{1}{a}\). \\
	
	\textbf{6. Chiusura del prodotto esterno \(\odot\):}
	\[
	\lambda \odot a = a^\lambda \quad \text{è un elemento di } \mathbb{R}^+ \text{ per ogni } \lambda \in \mathbb{R}, a \in \mathbb{R}^+.
	\]
	Un numero positivo elevato a una potenza reale è ancora un numero positivo, quindi questa proprietà è soddisfatta. \\
	
\textbf{	7. Compatibilità tra \(\odot\) e \(\oplus\):}
	\[
	\lambda \odot (a \oplus b) = \lambda \odot (a \cdot b) = (a \cdot b)^\lambda = a^\lambda \cdot b^\lambda = (\lambda \odot a) \oplus (\lambda \odot b).
	\]
	
\textbf{	8. Elemento neutro rispetto a \(\odot\):}
	\[
	1 \odot a = a^1 = a \quad \text{per ogni } a \in \mathbb{R}^+.
	\]
	
\textbf{	9. Distributività di \(\odot\) rispetto a \(\oplus\):}
	\[
	(\lambda + \mu) \odot a = a^{\lambda + \mu} = a^\lambda \cdot a^\mu = (\lambda \odot a) \oplus (\mu \odot a).
	\]
	
	Quindi \((\mathbb{R}^+, \oplus, \odot)\) soddisfa tutte le proprietà di uno spazio vettoriale su \(\mathbb{R}\). \\
	 
	a2) Per quanto riguarda la dimensione di questo spazio vettoriale, notiamo che ogni elemento di \(\mathbb{R}^+\) può essere scritto come \(a^\lambda\), dove \(\lambda \in \mathbb{R}\). Poiché possiamo rappresentare ogni elemento di \(\mathbb{R}^+\) con una singola potenza, lo spazio vettoriale \((\mathbb{R}^+, \oplus, \odot)\) ha dimensione 1. \\
	
	
	
	b) Definiamo la mappa \(\phi: \mathbb{R}^+ \to \mathbb{R}\) data da:
	\[
	\phi(a) = \log(a) \quad \text{per ogni } a \in \mathbb{R}^+.
	\]
	
	Verifichiamo che \(\phi\) sia un isomorfismo:
	
	\textbf{1. Linearità:}
	\[
	\phi(a \oplus b) = \phi(a \cdot b) = \log(a \cdot b) = \log(a) + \log(b) = \phi(a) + \phi(b).
	\]
	
	\textbf{2. Compatibilità con il prodotto scalare:}
	\[
	\phi(\lambda \odot a) = \phi(a^\lambda) = \lambda \log(a) = \lambda \cdot \phi(a).
	\]

	\textbf{3. Biunivocità:}
	\(\phi\) è iniettiva perché \(\log(a) = \log(b)\) implica \(a = b\), ed è suriettiva perché per ogni \(y \in \mathbb{R}\), esiste un \(a \in \mathbb{R}^+\) tale che \(\log(a) = y\).
	
	Quindi \(\phi\) è un isomorfismo tra \((\mathbb{R}^+, \oplus, \odot)\) e \(\mathbb{R}\), che è uno spazio vettoriale di dimensione 1.
	
	Pertanto, \((\mathbb{R}^+, \oplus, \odot)\) è isomorfo a \(\mathbb{R}\).
