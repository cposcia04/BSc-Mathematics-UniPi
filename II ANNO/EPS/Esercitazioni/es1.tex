\section{Esercitazione 06/03/2025}
\subsubsection{Esercizio 1}
Tirando un dado non truccato due volte, si descriva uno spazio degli esiti e una misura di probabilità $P$ per modellare il risultato di questo esperimento. Sia $A$ l’evento "\textit{il secondo lancio più grande del primo}". Calcolare la probabilià $P(A)$.
\\

\textit{Soluzione} \quad 

\subsubsection{Esercizio 2}
In un gioco il giocatore ed il banco lanciano entrambi per 10 volte una moneta equilibrata. Il giocatore vince solo se il numero di teste da lui ottenuto è maggiore strettamente del numero di teste ottenuto dal banco. Qual è la probabilit`a che il giocatore vinca?
\\

\textit{Soluzione} \quad 

\subsubsection{Esercizio 3}
Le $n$ cifre di un numero sono scelte in maniera casuale. Calcolare la probabilità che
(a) non appaia il 3;
(b) non appaiono né il 4 né il 7; (c) appaia almeno un 5.
Scrivere poi un’espressione per la probabilità che nel numero il 3 appaia prima del 4.
\\

\textit{Soluzione} \quad 

\subsubsection{Esercizio 6} Si scrivono su 11 foglietti di carta le lettere della parola ABRACADABRA, una per foglietto, e le si pongono in un contenitore. Si estraggono poi, a caso, i foglietti. Qual è la probabilità che le lettere, nell’ordine estratto, diano di nuovo la stessa parola?
\\

\textit{Soluzione} \quad 
$\Omega =\{\omega_1,...,\omega_{11} \; | \; \omega_i \in \Omega, \omega_i \ne \omega_j , i\ne j\}$ 
\\

$\Omega_i = \{A_1,A_2,A_3,A_4,A_5,B_1,B_2,C_1,D_1,R_1,R_2\}$ 
\\

$P(\omega)=\frac{1}{|\Omega|}=\frac{1}{11!}$
\\

$P(\{ABRACADABRA\})=\frac{5! \;2!\; 2!}{11!}$
