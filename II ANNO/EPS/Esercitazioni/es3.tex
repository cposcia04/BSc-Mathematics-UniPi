\section{Foglio 3}
\subsection{Esercizio 1}
In un contenitore ci sono 100 palline numerate da 1 a 100. Le estraiamo una dopo l’altra senza reinserimento. \\
(a) Qual è la probabilità di ottenere nelle prime 10 estrazioni solo numeri $\leq  75$? \\
(b) Qual è la probabilità che le prime 15 palline estratte portino tutte un numero dispari? \\
(c) Qual è la probabilità che, tra le prime 5 palline estratte, ce ne siano esattamente 2 con numeri
dispari?
\\

\textit{Soluzione}

\subsection{Esercizio 2}
Una moneta viene lanciata $2n$ volte. Sia $t_n$ la probabilità che il numero di teste sia maggiore del numero di croci, e $u_n$ la probabilità che il numero di teste sia pari al numero di croci.\\
(a) Calcolare il limite di $u_n$ per $n \to +\infty$. \\
(b) Determinare il limite di $t_n$.
\\

\textit{Soluzione}

\subsection{Esercizio 3}
Il numero di telefonate $X $ che arrivano ad una segreteria telefonica di un ufficio ogni 9 minuti è
distribuito $X \sim \text{Pois}(6)$. Calcolare \\
(a) la probabilità che arrivino almeno 5 chiamate in 9 minuti; \\
(b) la probabilità che non arrivi nessuna chiamata tra le ore 9:00 e le ore 9:09
\\

\textit{Soluzione}

\subsection{Esercizio 4}
(Distribuzione multinomiale $\mathcal M(n, p_1 ,...,p_k)$). Consideriamo un esperimento di $n$ prove indipendenti ove ogni prova può avere uno solo tra $k$ possibili risultati, ognuno dei quali ha probabilità rispettivamente $p_1 ,p_2 ,...,p_k$ (e $p_1 +...+p_k = 1$) di accadere. Calcolare la probabilità $p(n_1 ,n_2 ,...,n_k )$ che si abbiano $n_1$ esiti del tipo 1, $n_2$ esiti del tipo $2, ... , n_k$ esiti del tipo $k$, al variare delle $k$-uple ($n_1 ,...,n_k $) per cui $n_1 +...+n_k =n$.
\\

\textit{Soluzione}

\subsection{Esercizio 5}
Consideriamo un esperimento a prove ripetute indipendenti con probabilità di successo $p$. Determinare la probabilità che \\
(a) il primo successo avvenga alla prova $k$; \\
(b) il primo successo avvenga dopo almeno $k$ prove; \\
(c) il primo successo avvenga prima della $(k + 1)$-esima prova; \\
(d) il primo successo avvenga in una prova dispari; \\
(e) il primo successo non avvenga mai.
\\

\textit{Soluzione}

\subsection{Esercizio 6}
Un generatore di numeri casuali produce una successione di terne $(i, j, k)$, con $i, j, k \in \{0, 1, 2, . . . , 9\}$. \\
Indichiamo con $S$ l’evento “esce un tris” (ovvero una terna costituita da cifre tutte uguali). Calcolare la probabilità dei seguenti eventi: \\

(a) tra le prime 10 terne prodotte ci sono almeno due tris. 
(b) si devono produrre almeno 10 terne per ottenere due tris. 
(c) si devono produrre esattamente 40 terne per avere 3 tris (ovvero il terzo tris si ha esattamente alla
40-esima terna prodotta).
\\

\textit{Soluzione}

\subsection{Esercizio 7}
Un collezionista ha già raccolto 60 delle 100 figurine di un album. Egli acquista una busta contenente 24 figurine (tutte diverse), tra le quali naturalmente ve ne possono essere alcune che egli già possiede. Qual è la probabilità che tra le figurine appena comprate ve ne siano almeno 20 che già possiede?
\\

\textit{Soluzione}

\subsection{Esercizio 8}
Siano dati due esperimenti a prove ripetute indipendenti con probabilità di successo rispettivamente $p_1$
e $p_2$ che siano indipendenti tra loro. \\
(a) Qual è la probabilità che il primo successo del primo esperimento avvenga prima del primo successo
del secondo? \\
(b) Assumiamo ora che vengano fatte 5 prove per esperimento. Calcolare la probabilità che il numero
di successi nel primo gruppo sia maggiore o uguale al numero di successi nel secondo.
\\

\textit{Soluzione}

\subsection{Esercizio 9}
Consideriamo un’urna con $N$ palline, di cui $N_1$ sono rosse, mentre $N - N_1$ sono di colore diverso dal rosso. Consideriamo l’esperimento che consiste nell’estrazione senza rimpiazzo di $n \leq  N$ palline, in cui il “successo” è l’estrazione di una pallina rossa.
Sia $W_n$ la variabile aleatoria che conta il numero di palline rosse estratte tra le $n$. Determinare il range di $W_n$ e la funzione di probabilità $p_{W_n}$ associata.
\\

\textit{Soluzione}

\subsection{Esercizio 10} Sia $X$ una variabile aleatoria discreta a valori interi positivi; si dice che $X$ gode della proprietà di
perdita di memoria se:
\[\mathbb{P}(X > n + k \mid X > k) = \mathbb{P}(X>n) \quad \forall n,k \in \mathbb{N}^*\]
(a) Si mostri che se $X \sim \text{Geom}(p)$, ossia $\mathbb{P}(X=k)=p(1-p)^{k-1}, \quad k=1,2,...$ allora $X$ soddisfa la proprietà di perdita di memoria. \\
(b) Vale anche il viceversa, ovvero: se $X$ è una variabile discreta a valori interi positivi che soddisfa
la proprietà di perdita di memoria, allora $X \sim \text{Geom}(p) $per un qualche $p \in (0, 1)$.
\\

\textit{Soluzione}

\subsection{Esercizio 11}
(Urna di Polya) Si consideri un’urna che contiene inizialmente due palline, una rossa e una verde. Si estrare una pallina, quindi la si reimmette nell’urna, aggiungendone poi un’altra dello stesso colore di quella estratta. Si itera quindi questa procedura di estrazione/reimmissione. Sia $X_n$  il numero di palline rosse presenti nell’urna dopo $n$ iterazioni. Calcolare la legge di $X_n$ (quindi la sua densità discreta $p(k) = \mathbb{P}(X = k) $ al variare di $k$) per ogni $N \in \mathbb{N}$.
\\

\textit{Soluzione}

\subsection{Esercizio 12}
Ci sono sei tazzine da caffè con corrispondenti piattini. Due sono di colore bianco (b), due rosse (r) e due oro (o). Disponiamo i piattini in ordine sul tavolo nella sequenza \textit{bbrroo}. Poi arriva una persona non vedente e dispone le tazzine a caso sui piattini. Sia M il numero di tazzine il cui colore corrisponde a quello del piattino sul quale sono state disposte. \\
(a) Calcolare $\mathbb{P}(M = 4)$. \\ (b) (Opzionale) Calcolare la distribuzione della variabile aleatoria M.
\\

\textit{Soluzione}



